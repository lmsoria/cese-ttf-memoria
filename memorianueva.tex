%----------------------------------------------------------------------------------------
% Preambulo y Configuración
%----------------------------------------------------------------------------------------

\documentclass[
    11pt,
    spanish,
    singlespacing,
    parskip,
    headsepline,
    bookmarks=true,
    unicode=true,
    pdftoolbar=true,
    pdfmenubar=true,
    pdffitwindow=false,
    colorlinks=true,
    linkcolor=blue,
    citecolor=blue,
    urlcolor=blue
]{MastersDoctoralThesis}

\usepackage[utf8]{inputenc} % Codificación de entrada UTF-8
\usepackage[T1]{fontenc}    % Codificación de salida para caracteres especiales
\usepackage{graphicx}       % Manejo de gráficos
\usepackage{eso-pic}        % Permite agregar fondos
\usepackage{hyperref}       % Manejo de hipervínculos y marcadores

% Redefinición de caracteres problemáticos en marcadores
\hypersetup{
    pdftitle={Título del Documento},
    pdfauthor={Autor del Documento},
    pdfkeywords={Sistemas Embebidos, Internet de las Cosas, Inteligencia Artificial},
    pdfstartview={FitH},
    unicode=true,
    colorlinks=true,
    linkcolor=blue,
    citecolor=blue,
    urlcolor=blue
}

\pdfstringdefDisableCommands{%
  \def\texttt#1{#1}%
  \def\textbf#1{#1}%
  \def\textit#1{#1}%
  \def\"{\"}%
  \def\~{~}%
  \def\'{'}%
  \def\^{}%
  \def\textunderscore{\_} % Manejo del subrayado en marcadores
}


% Definir comandos requeridos por la clase
\newcommand{\degreename}{Maestría en Ciencias} % Cambia según tu título
\newcommand{\univname}{Universidad Nacional de Ejemplo} % Cambia según tu universidad
\newcommand{\keywordnames}{Palabras clave:}
%----------------------------------------------------------------------------------------
% Documento Principal
%----------------------------------------------------------------------------------------

\begin{document}

% Configuración de la portada
\posgrado{Carrera / Maestría}
\keywords{Sistemas Embebidos, Internet de las Cosas, Inteligencia Artificial}

% Incluir la portada desde un archivo separado
\include{portada}

% Configuración del contenido preliminar
\frontmatter % Usar numeración romana para las páginas preliminares
\pagestyle{plain} % Estilo de encabezado simple

%----------------------------------------------------------------------------------------
% Resumen
%----------------------------------------------------------------------------------------

\begin{abstract}
\addchaptertocentry{\abstractname} % Agregar resumen al índice
En esta memoria se presenta el diseño y la construcción de un equipo con cuatro entradas y ocho salidas para el intercambio de mensajes entre instrumentos musicales. El desarrollo integra hardware, firmware y una aplicación de escritorio que permite configurar y supervisar el equipo de manera sencilla. El sistema posibilita definir rutas entre entradas y salidas, aplicar filtros de mensajes y guardar configuraciones para su uso posterior. Opera con baja latencia, característica que resulta especialmente útil en actuaciones en vivo y entornos de estudio, circunstancias en las que se requiere una respuesta rápida y confiable.

Para su realización se aplicaron conocimientos de electrónica, desarrollo de software para microcontroladores, diseño de protocolos de comunicación, modelado mediante máquinas de estados finitos y ensayos de validación. El trabajo se desarrolló como una iniciativa personal con potencial salida al mercado, orientada a necesidades de la comunidad argentina de músicos y productores, con énfasis en la confiabilidad, desempeño y facilidad de uso.
\end{abstract}

%----------------------------------------------------------------------------------------
% Agradecimientos
%----------------------------------------------------------------------------------------

\begin{acknowledgements}
\vspace{1.5cm}
Esta sección es para agradecimientos personales y es totalmente \textbf{OPCIONAL}.
\end{acknowledgements}

%----------------------------------------------------------------------------------------
% Índice
%----------------------------------------------------------------------------------------

\tableofcontents
\listoffigures
\listoftables

%----------------------------------------------------------------------------------------
% Dedicatoria
%----------------------------------------------------------------------------------------

\dedicatory{\textbf{Dedicado a... [OPCIONAL]}}

%----------------------------------------------------------------------------------------
% Capítulos
%----------------------------------------------------------------------------------------

\mainmatter % Iniciar numeración numérica para el contenido principal
\pagestyle{thesis} % Estilo de encabezado de tesis

% Incluir capítulos desde archivos separados
% Chapter 1

\chapter{Introducción general} % Main chapter title

\label{Chapter1} % For referencing the chapter elsewhere, use \ref{Chapter1}
\label{IntroGeneral}

Este capítulo introduce la problemática de la interconexión de instrumentos musicales electrónicos. Luego se analizan algunas alternativas comerciales y se describe la motivación por la que se realizó este trabajo. Finalmente, se presentan los objetivos y alcances del trabajo realizado.


%----------------------------------------------------------------------------------------

% Define some commands to keep the formatting separated from the content
\newcommand{\keyword}[1]{\textbf{#1}}
\newcommand{\tabhead}[1]{\textbf{#1}}
\newcommand{\code}[1]{\texttt{#1}}
\newcommand{\file}[1]{\texttt{\bfseries#1}}
\newcommand{\option}[1]{\texttt{\itshape#1}}
\newcommand{\grados}{$^{\circ}$}

%----------------------------------------------------------------------------------------

%\section{Introducción}

%----------------------------------------------------------------------------------------
\section{Introducción a la problemática} % 1 página

Es frecuente encontrar en estudios de grabación y/o producción musical una variedad de instrumentos musicales electrónicos, principalmente sintetizadores o unidades de efectos.

Estos equipos se comunican mediante el estándar MIDI (\textit{Musical Instrument Digital Interface}, Interfaz Digital de Instrumentos Musicales). Este define un mecanismo de comunicación que permite la transferencia de información entre los instrumentos, computadoras y otros dispositivos relacionados con la producción musical.

Para garantizar una correcta interoperabilidad, MIDI no solo define el protocolo (es decir, el formato de los mensajes), sino también la conexión física entre ellos mediante la especificación de los cables y conectores que deben usarse.

En la especificación original, la \textit{MIDI Manufacturers Association} (MMA, Asociación de Fabricantes MIDI) adoptó el conector DIN de 5 pines a 180° como interfaz física estandarizada para interconectar equipos \cite{mma:midi_spec}. En la figura \ref{fig:din5} se muestra, a modo de ejemplo, el panel trasero de un sintetizador, donde se observan dos conectores etiquetados como MIDI OUT y MIDI IN.

\begin{figure}[ht]
	\centering
	\includegraphics[width=0.55\textwidth]{./Figures/din-connector.jpg}
	\caption{Ejemplo de cable y conectores DIN-5 pertenecientes al panel trasero de un sintetizador.}
	\label{fig:din5}
\end{figure}

A partir de lo expuesto anteriormente se infiere que la comunicación es unidireccional: se requiere de un conector de salida (MIDI OUT) para el envío de datos y otro de entrada (MIDI IN) para su recepción. El estándar también menciona un puerto adicional y opcional llamado MIDI THRU, que retransmite lo recibido en la entrada. Esto permite encadenar varios dispositivos entre sí, conformando una topología en cadena (\textit{daisy chain}). En la figura \ref{fig:daisy} se presenta un diagrama ilustrativo de este tipo de conexión.

\begin{figure}[ht]
	\centering
	\includegraphics[width=\textwidth]{./Figures/daisy-chain.png}
	\caption{Diagrama de una topología \textit{daisy chain}. Ésta requiere que el equipo disponga de un puerto MIDI THRU.}
	\label{fig:daisy}
\end{figure}

Al incrementar la cantidad de equipos se dificulta la interconexión entre ellos. En la figura \ref{fig:ad-hoc} se muestra una hipotética conexión de dos controladores con cuatro sintetizadores.

\begin{figure}[ht]
	\centering
	\includegraphics[width=\textwidth]{./Figures/ad-hoc.png}
	\caption{Ejemplo de una conexión hipotética entre dos controladores y cuatro sintetizadores. El secuenciador controlaría a los sintetizadores 2, 3 y 4, mientras que el controlador controlaría al sintetizador 1.}
	\label{fig:ad-hoc}
\end{figure}

Dado que la salida MIDI THRU replica lo que ingresa por MIDI IN, el secuenciador únicamente puede controlar la cadena de sintetizadores 2, 3 y 4, mientras que el controlador solo puede controlar el sintetizador 1. Para invertir este esquema de control es necesario reconfigurar el cableado, lo que puede resultar incómodo desde el punto de vista operativo e incluso propenso a errores.

Una solución a esta clase de problemas consiste en incorporar un equipo adicional llamado \textit{router} MIDI, que dispone de múltiples entradas y salidas independientes. Al interconectar los equipos a través de él se obtiene una topología en estrella, con el router como nodo central. En la figura \ref{fig:centralizada} se muestran los mismos equipos de la figura \ref{fig:ad-hoc} conectados a través del router.

\begin{figure}[ht]
	\centering
	\includegraphics[width=\textwidth]{./Figures/centralizada.png}
	\caption{Reconexión de los equipos de la figura \ref{fig:ad-hoc}, esta vez usando un router central.}
	\label{fig:centralizada}
\end{figure}

Se observa un cableado más directo e intuitivo, lo que reduce la probabilidad de errores. Además, tanto el controlador como el secuenciador pueden controlar a cualquiera de los cuatro sintetizadores, lo que aporta mayor flexibilidad de uso.
%----------------------------------------------------------------------------------------

\section{Estado del arte}

Hoy en día existen diversas soluciones comerciales en el mercado global. A la hora de elegir un equipo se tienen en cuenta (entre otras cosas) las siguientes características:
\begin{itemize}
	\item Cantidad de entradas y salidas.
	\item Capacidad de direccionamiento de las entradas.
	\item Filtrado de mensajes MIDI.
	\item Configuración mediante PC o \textit{stand-alone}.
\end{itemize}

En esta sección se presenta una lista general (pero no exhaustiva) de algunas de las alternativas más populares.

\subsection{Kenton THRU-12}
Desarrollado por la firma británica Kenton, este modelo posee una entrada que es replicada a sus doce salidas. A los equipos con esta característica se los conoce también como \textit{thru boxes}. En la figura \ref{fig:kenton} se muestra una fotografía del producto.

\begin{figure}[ht]
	\centering
	\includegraphics[width=0.5\textwidth]{./Figures/kenton.png}
	\caption{Kenton THRU-12. Es una \textit{thru box} de una entrada y doce salidas. Imagen tomada de la página oficial del fabricante\protect\footnotemark.}
	\label{fig:kenton}
\end{figure}

\footnotetext{Imagen tomada de \url{https://kentonuk.com/product/thru-12/}}

\subsection{MIDI Solutions Quadra Merge}
Desarrollado por la firma canadiense MIDI Solutions, este modelo posee cuatro entradas que se combinan y son redirigidas a dos salidas. Los equipos que fusionan varias entradas en una o más salidas son conocidos como \textit{merge boxes}. En la figura \ref{fig:midi-solutions} se muestra una fotografía del producto.

\begin{figure}[ht]
	\centering
	\includegraphics[width=0.5\textwidth]{./Figures/midi-solutions.jpg}
	\caption{MIDI Solutions Quadra Merge. Es una \textit{merge box} de cuatro entradas y dos salidas. Imagen tomada de la página oficial del fabricante\protect\footnotemark.}
	\label{fig:midi-solutions}
\end{figure}

\footnotetext{Imagen tomada de \url{https://midisolutions.com/prodqmr.htm}}

\subsection{Conductive Labs MRCC}
Esta solución de la firma estadounidense Conductive Labs posee once entradas y diecisiete salidas. También ofrece opciones avanzadas de ``routeo'' y filtrado de mensajes, configurables a través de una pantalla integrada. En la figura \ref{fig:mrcc} se muestra una fotografía del producto.

\begin{figure}[ht]
	\centering
	\includegraphics[width=0.8\textwidth]{./Figures/mrcc.png}
	\caption{Conductive Labs MRCC. Es un router MIDI de once entradas y diecisiete salidas, con soporte de routeo y filtrado de mensajes. Imagen tomada de la página oficial del fabricante\protect\footnotemark.}
	\label{fig:mrcc}
\end{figure}

\footnotetext{Imagen tomada de \url{https://conductivelabs.com/mrcc/?v=c582dec943ff}}

\subsection{iConnectivity mioXL}
Esta solución de la firma estadounidense iConnectivity posee ocho entradas y doce salidas. También soporta MIDI sobre Ethernet y USB. Al igual que el producto anterior, también ofrece opciones avanzadas de ``routeo'' y filtrado de mensajes configurables desde una aplicación de escritorio. En la figura \ref{fig:mioxl} se muestra una fotografía del producto.

\begin{figure}[ht]
	\centering
	\includegraphics[width=0.9\textwidth]{./Figures/mioxl.jpg}
	\caption{iConnectivity mioXL. Es un router MIDI de ocho entradas y doce salidas, con soporte de routeo y filtrado de mensajes. Imagen tomada de la página oficial del fabricante\protect\footnotemark.}
	\label{fig:mioxl}
\end{figure}

\footnotetext{Imagen tomada de \url{https://www.iconnectivity.com/mioxl}}


\subsection{Comparativa}
Luego de mostrar algunas alternativas comerciales, se recopila en la tabla \ref{tab:comparativa-modelos} la lista de características analizada de cada una de ellas, agregando también el costo del producto en dólares estadounidenses.

\begin{table}[h]
	\centering
	\caption[Comparativa entre alternativas comerciales.]{Comparativa de las alternativas comerciales expuestas.}
	\label{tab:comparativa-modelos}
	\resizebox{\textwidth}{!}
	{
		\begin{tabular}{l c c c c c c}
			\toprule
			\textbf{Modelo} & \textbf{Entradas} & \textbf{Salidas} & \textbf{Routeo} & \textbf{Filtrado} & \textbf{Configuración} & \textbf{Costo} \\
			\midrule
			Kenton THRU-12 & 1 & 12 & No & No & - & \$ 150 \\
			MIDI Solutions Quadra Merge & 4 & 2 & No & No & - & \$ 175 \\
			Conductive Labs MRCC & 11 & 17 & Si & Si & \textit{Stand-alone} & \$ 439 \\
			iConnectivity mioXL & 8 & 12 & Si & Si & Aplicación & \$ 500 \\
			\bottomrule
			\hline
		\end{tabular}
	}
\end{table}

%----------------------------------------------------------------------------------------

\section{Motivación del trabajo}

En la sección anterior se mostraron algunos ejemplos de alternativas comerciales, con distintos niveles de flexibilidad. No obstante, su adquisición en el contexto nacional enfrenta barreras asociadas a costos finales, logística de importación y soporte posventa. En contraste, la oferta local se concentra en dispositivos de baja complejidad (principalmente \textit{thru boxes}), adecuados para replicación de señales pero insuficientes cuando se requiere ``routeo'' selectivo entre entradas y salidas o filtrado por tipo de mensaje.

Es así como se planteó el desafío de desarrollar una versión local de menor costo que:
\begin{itemize}
	\item Soporte múltiples entradas y salidas DIN.
	\item Soporte filtrado de mensajes.
	\item Permita configurar el direccionamiento entre entradas y salidas.
\end{itemize}

Finalmente, cabe destacar que pese a que la motivación de este trabajo fue personal, se realizaron encuestas en foros de entusiastas argentinos, con la intención de orientar el desarrollo de forma tal que el trabajo tenga potencial de venta en el futuro. Esto garantizaría poder cubrir un nicho que actualmente no está cubierto.

%----------------------------------------------------------------------------------------

\section{Objetivos y alcance}

Se planteó como objetivo de este trabajo el desarrollo integral del prototipo de un \textit{router} MIDI de cuatro entradas y ocho salidas. Esto implicó cubrir los siguientes puntos:
\begin{itemize}
	\item Diseñar y ensamblar el circuito electrónico del equipo.
	\item Desarrollar el firmware para brindar la funcionalidad deseada.
	\item Desarrollar una aplicación para controlar al equipo.
	\item Diseñar un gabinete donde se alojen las placas.
\end{itemize}

Quedó fuera del alcance de este trabajo:
\begin{itemize}
	\item El diseño del hardware orientado a cumplir con certificaciones regulatorias.
	\item La planificación de una manufactura a escala.
	\item La optimización de costos para producción en serie.
\end{itemize}

%----------------------------------------------------------------------------------------

% Chapter 2
\chapter{Introducción específica}

\label{Chapter2}

%----------------------------------------------------------------------------------------

En este capítulo se describe brevemente cómo se codifican los mensajes MIDI de acuerdo al estándar, y posteriormente se listan las herramientas utilizadas para el desarrollo del firmware, el software y el hardware del trabajo.

%----------------------------------------------------------------------------------------

\section{Formato de mensajes MIDI} % 1 página

El estándar MIDI está basado en el envío de datos a través de un puerto serie opto-aislado que opera a una velocidad de 31250 baudios. Para codificar los mensajes, en primer lugar se hace la distinción entre dos tipos de bytes: de Estado (\textit{Status Byte}) y de Datos (\textit{Data Byte}). En la figura \ref{fig:midi-byte} se ilustra esta diferencia.

\begin{figure}[ht]
	\centering
	\includegraphics[width=0.5\textwidth]{./Figures/midi-byte.png}
	\caption{Tipos de ``bytes'' según el estándar MIDI.}
	\label{fig:midi-byte}
\end{figure}

Un byte de datos es aquel que tiene su MSB (\textit{Most Significant Bit}, bit más significativo) igual a cero. Es decir que su rango está entre \texttt{0x00} (0 decimal) y \texttt{0x0F} (127 decimal). Por el contrario, un byte de estado tiene su MSB igual a uno, por lo que su rango oscila entre \texttt{0x80} (128 decimal) y \texttt{0xFF} (255  decimal).

El propósito de un byte de estado es el de actuar como encabezado, identificando el tipo de mensaje MIDI. Este consistirá de un byte de estado, seguido de:
\begin{enumerate}
	\item Ningún byte de dato.
	\item Un byte de dato.
	\item Dos bytes de dato.
	\item Múltiple bytes de dato.
\end{enumerate}

En la figura \ref{fig:structure-midi-message} se ilustran estos cuatro posible casos.

\begin{figure}[ht]
	\centering
	\includegraphics[width=\textwidth]{./Figures/structure-midi-message.png}
	\caption{}
	\label{fig:structure-midi-message}
\end{figure}
%----------------------------------------------------------------------------------------

%----------------------------------------------------------------------------------------

\section{Hardware de desarrollo} % 1 página

%----------------------------------------------------------------------------------------

%----------------------------------------------------------------------------------------

\section{Herramientas de desarrollo de firmware} % 1 página

%----------------------------------------------------------------------------------------

%----------------------------------------------------------------------------------------

\section{Framework para el desarrollo del software} % 1 página

%----------------------------------------------------------------------------------------
\chapter{Diseño e implementación} % Main chapter title

\label{Chapter3} % Change X to a consecutive number; for referencing this chapter elsewhere, use \ref{ChapterX}

En este capítulo se detalla el proceso de desarrollo del equipo, incluyendo la arquitectura de la solución como sistema, el firmware embebido, la aplicación de escritorio y el hardware que constituye el router.
\definecolor{mygreen}{rgb}{0,0.6,0}
\definecolor{mygray}{rgb}{0.5,0.5,0.5}
\definecolor{mymauve}{rgb}{0.58,0,0.82}

%%%%%%%%%%%%%%%%%%%%%%%%%%%%%%%%%%%%%%%%%%%%%%%%%%%%%%%%%%%%%%%%%%%%%%%%%%%%%
% parámetros para configurar el formato del código en los entornos lstlisting
%%%%%%%%%%%%%%%%%%%%%%%%%%%%%%%%%%%%%%%%%%%%%%%%%%%%%%%%%%%%%%%%%%%%%%%%%%%%%
\lstset{ %
  backgroundcolor=\color{white},   % choose the background color; you must add \usepackage{color} or \usepackage{xcolor}
  basicstyle=\footnotesize,        % the size of the fonts that are used for the code
  breakatwhitespace=false,         % sets if automatic breaks should only happen at whitespace
  breaklines=true,                 % sets automatic line breaking
  captionpos=b,                    % sets the caption-position to bottom
  commentstyle=\color{mygreen},    % comment style
  deletekeywords={...},            % if you want to delete keywords from the given language
  %escapeinside={\%*}{*)},          % if you want to add LaTeX within your code
  %extendedchars=true,              % lets you use non-ASCII characters; for 8-bits encodings only, does not work with UTF-8
  %frame=single,	                % adds a frame around the code
  keepspaces=true,                 % keeps spaces in text, useful for keeping indentation of code (possibly needs columns=flexible)
  keywordstyle=\color{blue},       % keyword style
  language=[ANSI]C,                % the language of the code
  %otherkeywords={*,...},           % if you want to add more keywords to the set
  numbers=left,                    % where to put the line-numbers; possible values are (none, left, right)
  numbersep=5pt,                   % how far the line-numbers are from the code
  numberstyle=\tiny\color{mygray}, % the style that is used for the line-numbers
  rulecolor=\color{black},         % if not set, the frame-color may be changed on line-breaks within not-black text (e.g. comments (green here))
  showspaces=false,                % show spaces everywhere adding particular underscores; it overrides 'showstringspaces'
  showstringspaces=false,          % underline spaces within strings only
  showtabs=false,                  % show tabs within strings adding particular underscores
  stepnumber=1,                    % the step between two line-numbers. If it's 1, each line will be numbered
  stringstyle=\color{mymauve},     % string literal style
  tabsize=2,	                   % sets default tabsize to 2 spaces
  title=\lstname,                  % show the filename of files included with \lstinputlisting; also try caption instead of title
  morecomment=[s]{/*}{*/}
}


%----------------------------------------------------------------------------------------

\section{Diagrama en bloques del sistema} \label{diagrama-bloques-sistema}
El trabajo desarrollado consiste en dos grandes bloques. Por un lado se encuentra el firmware (que corre en la plataforma de hardware), y por el otro lado software que corre nativamente en una computadora y se comunica con el firmware. En la figura \ref{fig:sys-arch} se ilustra un diagrama en bloques de la solución implementada.

\begin{figure}[ht]
	\centering
	\includegraphics[width=0.75\textwidth]{./Figures/system-arch.png}
	\caption{Diagrama en bloques del trabajo realizado.}
	\label{fig:sys-arch}
\end{figure}

El firmware se encarga de realizar íntegramente el procesamiento del flujo de datos (mensajes MIDI), mientras que el software actúa como interfaz de configuración del comportamiento del equipo. La comunicación entre ambas es mediante un puerto serie virtual, emulado por la interfaz USB del microcontrolador. A través de una serie de comandos, se modifican las propiedades de una estructura que modela el comportamiento de la aplicación (\texttt{Settings}). En la figura \ref{fig:app-settings} se muestra un diagrama de clases listando todas las estructuras que componen a este modelo.

\begin{figure}[ht]
	\centering
	\includegraphics[width=0.75\textwidth]{./Figures/app-settings.png}
	\caption{Diagrama en clases de las estructuras de datos principales de la aplicación.}
	\label{fig:app-settings}
\end{figure}

Recorriendo el diagrama de menor a mayor jerarquía, se encuentran las siguientes estructuras:
\begin{itemize}
	\item \texttt{Version}: Se utiliza para versionar la imagen de firmware.
	\item \texttt{SystemFilters}: Representa mediante un \texttt{std::bitset} \cite{cppreference:bitset} de 11 bits si los mensajes de sistema deben ser filtrados o no (un bit por cada tipo de mensaje).
	\item \texttt{ChannelFilters}: Cumple la misma funcionalidad que \texttt{SystemFilters}, con la salvedad de que, como los mensajes de canal tienen un canal asociado, esta estructura es un arreglo de dieciséis \texttt{std::bitset} de 7 bits.
	\item \texttt{RouterSettings}: El ruteo entre entradas y salidas también se modela con un \texttt{std::bitset}. Dado que el sistema posee 4 entradas y 8 salidas, esta estructura es un arreglo de cuatro \texttt{std::bitset} de 8 bits.
	\item \texttt{IOSettings}: Esta estructura encapsula todas las propiedades que se pueden modificar en las entradas y salidas: un nombre, la configuración de filtros y si se encuentra habilitada o no.
	\item \texttt{Preset}: Contiene la configuración de todas las entradas y salidas, así como la configuración de ruteo entre ellas. Además se le agrega un nombre para que el usuario también pueda identificarla.
	\item \texttt{Settings}: Contiene el estado de los dieciséis preset, el número de preset activo, la versión y un identificador único que se utiliza para operaciones de serialización.
\end{itemize}
%----------------------------------------------------------------------------------------

%----------------------------------------------------------------------------------------

\section{Desarrollo del firmware} \label{desarrollo-firmware}

En esta sección se describe la arquitectura del firmware y los componentes que lo integran.

\subsection{Arquitectura del firmware} \label{arquitectura-firmware}
Para desarrollar la aplicación, se segmentó la funcionalidad en distintas capas con diferentes niveles de abstracción. En la figura \ref{fig:fw-layers} se ilustra la segmentación realizada.

\begin{figure}[ht]
	\centering
	\includegraphics[width=0.55\textwidth]{./Figures/fw-layers.png}
	\caption{Arquitectura de capas utilizada.}
	\label{fig:fw-layers}
\end{figure}

A continuación se detalla la funcionalidad de cada una de ellas:
\begin{itemize}
	\item \texttt{Hardware}: No es parte del software. Se la incluye en el diagrama para poner en contexto a las siguientes capas.
	\item \texttt{ST HAL}: Es la biblioteca HAL ofrecida por el fabricante del microcontrolador \cite{st:um1725}. Se la usa para configurar el reloj del sistema y los periféricos utilizados. En esta capa se contempla tambien la biblioteca USB desarrollada por ST.
	\item \texttt{freeRTOS}: Es la biblioteca de freeRTOS, también desarrollada por terceros \cite{freertos:rtos-book}.
	\item \texttt{os}: Es un wrapper en C++ de primitivas de kernel tales como colas, tareas o mutex, entre otras. Su funcionalidad es proveer una capa de abstracción sobre el sistema operativo utilizado, así como ofrecer más niveles de personalización, facilidad de uso y seguridad sobre las primitivas encapsuladas.
	\item \texttt{hal}: Es un wrapper en C++ de los periféricos utilizados, tales como UART, GPIO, USB, etc. Al igual que \texttt{os} busca desacoplar el periférico de la plataforma utilizada.
	\item \texttt{svc}: Capa de servicios de la aplicación. Son módulos de mayor nivel que engloban tareas específicas, más relacionadas con los requerimientos de la aplicación.
	\item \texttt{app}: Capa de aplicación. Es la encargada de instanciar todos los servicios de modo de cumplir con los requerimientos del firmware. También define los tipos de datos mencionados en la sección \ref{diagrama-bloques-sistema}.
\end{itemize}

\newpage

En base al esquema de capas propuesto, en la figura \ref{fig:fw-arch} se observa un diagrama en alto nivel de los componentes que integran a la aplicación. En secciones posteriores se realizará un análisis más exhaustivo.

\begin{figure}[ht]
	\centering
	\includegraphics[width=\textwidth]{./Figures/fw-arch.png}
	\caption{Diagrama en bloques de la aplicación embebida.}
	\label{fig:fw-arch}
\end{figure}

La aplicación consiste en cinco tipos de servicios o sistemas interactuando entre sí:
\begin{enumerate}
	\item \texttt{svc::dial}: Es el encargado de monitorear el estado del encoder rotativo y reportar interacciones con el usuario.
	\item \texttt{svc::led}: Controla los LEDs del panel frontal, con la intención principal de indicar el número de preset activo.
	\item \texttt{svc::storage}: Se encarga de leer y escribir la configuración de la aplicación en la memoria interna del microcontrolador.
	\item \texttt{svc::comms}: Gestiona la comunicación con la computadora. Es la encargada de serializar y deserializar los comandos establecidos por el protocolo diseñado para la comunicación entre el firmware y el software.
	\item \texttt{svc::io}: Es el que controla la recepción y transmisión de mensajes MIDI entre las entradas y las salidas. Esto incluye el ruteo y filtrado de mensajes
\end{enumerate}

Cada uno de estos servicios tiene una serie de dependencias que necesita para operar correctamente. La aplicación es la encargada de instanciarlas e inyectarlas.
Por otro lado, la comunicación entre los servicios y la aplicación se hace mediante el patrón de Objeto Activo \cite{lavender-schmidt:active-object}, en donde una tarea monitorea el estado de una cola de eventos que son procesados de manera diferida. Estos eventos son estructuras de datos generadas asincrónicamente por los servicios, y pueden contener o no información asociada.

\newpage

\subsection{Diseño del servicio de Dial}
El servicio \texttt{svc::dial} se encarga de monitorear un encoder rotativo con pulsador. En la figura \ref{fig:rotary-encoder} se muestra una fotografía de un encoder típico, montado en una placa tipo breakout para prototipos.

\begin{figure}[ht]
	\centering
	\includegraphics[width=0.25\textwidth]{./Figures/rotary-encoder.jpg}
	\caption{Encoder rotativo con pulsador\protect\footnotemark.}
	\label{fig:rotary-encoder}
\end{figure}

\footnotetext{Imagen tomada de \url{https://thepihut.com/cdn/shop/articles/DSC_0700.jpg}.}

Estos encoders poseen tres señales: una es el pulsador, y las otras dos, denominadas \texttt{CLK} y \texttt{DT}, generan una onda cuadrada desfasada 90 grados cuando el encoder rota. La fase relativa entre ambas señales determina el sentido de giro del dial.

En la figura \ref{fig:svc-dial} se ilustra un diagrama en bloques del servicio \texttt{svc::dial}. Este consiste en una tarea que muestrea periódicamente tres GPIOs configurados como entradas, aplicando mecanismos de antirrebote y procesamiento de la señal en cuadratura para determinar el estado del dial.

\begin{figure}[ht]
	\centering
	\includegraphics[width=\textwidth]{./Figures/svc-dial.png}
	\caption{Diagrama en bloques del servicio de Dial.}
	\label{fig:svc-dial}
\end{figure}

Cuando se detecta alguna actividad por parte del usuario, los siguientes eventos son enviados a la aplicación principal:
\begin{itemize}
	\item \texttt{ButtonPressed}: El botón ha sido apretado una sola vez.
	\item \texttt{ButtonDoublePressed}: El botón ha sido apretado dos veces en una ventana de tiempo corta (configurable a través de una constante de compilación).
	\item \texttt{ButtonLongPressed}: El botón ha sido apretado y retenido por un intervalo de tiempo largo (configurable a través de una constante de compilación).
	\item \texttt{DialCW}: El encoder ha rotado en sentido horario.
	\item \texttt{DialCCW}: El encoder ha rotado en sentido antihorario.
\end{itemize}


\subsection{Diseño del servicio de LEDs}
El servicio \texttt{svc::led} está encargado de controlar los dieciséis LEDs del panel frontal. En la figura \ref{fig:svc-led} se ilustra un diagrama en bloque de la estructura del servicio.

\begin{figure}[ht]
	\centering
	\includegraphics[width=0.6\textwidth]{./Figures/svc-led.png}
	\caption{Diagrama en bloques del servicio de LEDs.}
	\label{fig:svc-led}
\end{figure}

El servicio consiste en una tarea que periódicamente refresca el estado de los LEDs. El uso de una tarea permite  realizar animaciones, definidas a través de una FSM (\textit{Finite State Machine}) como la de la figura \ref{fig:svc-led-fsm}.

\begin{figure}[ht]
	\centering
	\includegraphics[width=\textwidth]{./Figures/svc-led-fsm.png}
	\caption{Máquina de estados del servicio de LEDs.}
	\label{fig:svc-led-fsm}
\end{figure}

Se definieron tres animaciones:
\begin{enumerate}
	\item \texttt{Startup}: Ocurre cuando el sistema inicializa. Hace un barrido por todos los LEDs, comenzando y terminando por el LED correspondiente al preset activo. Cuando finaliza la animación se transiciona al estado \texttt{Steady}.
	\item \texttt{Steady}: Es el estado por defecto del sistema. Se mantiene encendido el LED correspondiente al preset activo. El servicio dispone de una función llamada \texttt{set\_active\_preset()} para poder actualizar el LED a encender.
	\item \texttt{Blink}: Este estado se utiliza para indicarle al usuario que un preset ha sido guardado como activo, invocando el método \texttt{confirm\_saved()}. La animación consiste en hacer un destello intermitente dos veces. Cuando finaliza, se transiciona al estado \texttt{Steady}.
\end{enumerate}

\newpage

\subsection{Diseño del servicio de almacenamiento}
La función de \texttt{svc::storage} es la de ofrecer una interfaz para leer y escribir la configuración del sistema, modelada mediante la estructura \texttt{app::Settings} mencionada en la sección \ref{diagrama-bloques-sistema}. En la figura \ref{fig:svc-storage} se ilustra un diagrama en bloques del servicio.

\begin{figure}[ht]
	\centering
	\includegraphics[width=0.6\textwidth]{./Figures/svc-storage.png}
	\caption{Diagrama en bloques del servicio de almacenamiento.}
	\label{fig:svc-storage}
\end{figure}

\subsection{Diseño del servicio de I/O}
El servicio \texttt{svc::io} es el núcleo de la aplicación, ya que es el encargado de procesar el flujo de mensajes MIDI desde las entradas hasta las salidas. En la figura \ref{fig:svc-io} se observa un diagrama en bloques del mismo.
\begin{figure}[ht]
	\centering
	\includegraphics[width=\textwidth]{./Figures/svc-io.png}
	\caption{Diagrama en bloques del servicio de I/O.}
	\label{fig:svc-io}
\end{figure}

Dado que hay UARTs que operan como entrada o como entrada-salida, el servicio de I/O admite distintas especializaciones, determinadas por el parámetro \texttt{Mode}, que puede ser \texttt{Input} (solo entrada), \texttt{Output} (solo salida) o \texttt{InputOutput} (entrada y salida). Si el servicio admite entrada, se instanciará una tarea que se encarga de procesar los mensajes provenientes de la UART. Por otra parte, si el servicio admite salida, se instanciará una tarea que se encarga de procesar los mensajes recibidos y transmitirlos por la UART.

El flujo de entrada-salida sigue un patrón de diseño conocido como segmentación de procesos. Este permite transformar datos de una representación a otra mientras se mueven secuencialmente desde un productor a un consumidor \cite{sommerville:ingenieria-sw}. Para ilustrar este concepto, en las figuras \ref{fig:svc-in-pipe} y \ref{fig:svc-out-pipe} se representan los flujos de datos en las tareas de recepción y transmisión de mensajes, respectivamente.

\begin{figure}[ht]
	\centering
	\includegraphics[width=\textwidth]{./Figures/svc-in-pipe.png}
	\caption{Flujo de datos para la recepción de mensajes MIDI.}
	\label{fig:svc-in-pipe}
\end{figure}

La tarea de recepción involucra más pasos, ya que es necesario convertir la información binaria recibida por la UART en mensajes MIDI. Para ello, se cuenta con una clase llamada \texttt{app::Parser} encargada de convertir una secuencia de bytes en una secuencia de mensajes MIDI. Una vez detectados los tipos de mensaje recibidos, se procede a filtrarlos. La clase \texttt{app::Filter} verifica en la configuración de usuario si el mensaje se encuentra bloqueado. En caso negativo, se transmite a la siguiente etapa, donde la clase \texttt{app::Router} consulta en la configuración a qué salidas debe propagarse. Como los procesos de recepción y transmisión se gestionan en tareas diferentes, y dado que más de una tarea de entrada puede escribir en una tarea de salida, se utilizan colas de mensajes como mecanismo de sincronización para enviar los mensajes MIDI desde las entradas hacia las salidas. FreeRTOS garantiza que estas estructuras son aptas para situaciones de concurrencia \cite{freertos:queues}.


El proceso de transmisión involucra menos pasos, tal como puede observarse en la figura \ref{fig:svc-out-pipe}. La tarea permanece bloqueada hasta que aparezcan mensajes en la cola. Ante la presencia de ellos se procede a consultar la configuración del filtro de la salida. Si el mensaje no debe ser bloqueado, se procede a transmitirlo por el puerto serie. Para ello es necesario serializar primero el mensaje, responsabilidad de la clase \texttt{app::Serializer}.
\begin{figure}[ht]
	\centering
	\includegraphics[width=\textwidth]{./Figures/svc-out-pipe.png}
	\caption{Flujo de datos para la transmisión de mensajes MIDI.}
	\label{fig:svc-out-pipe}
\end{figure}

Una vez que el mensaje fue serializado, se copia su contenido en el buffer circular de la UART para su posterior transmisión, dando por finalizado el proceso.

\subsection{Diseño del servicio de comunicación}

%----------------------------------------------------------------------------------------

%----------------------------------------------------------------------------------------

\section{Desarrollo del software} % 1 página

%----------------------------------------------------------------------------------------

%----------------------------------------------------------------------------------------

\section{Desarrollo del hardware} % 1 página

%----------------------------------------------------------------------------------------





% Chapter Template

\chapter{Ensayos y resultados} % Main chapter title

\label{Chapter4} % Change X to a consecutive number; for referencing this chapter elsewhere, use \ref{ChapterX}
Todos los capítulos deben comenzar con un breve párrafo introductorio que indique cuál es el contenido que se encontrará al leerlo.  La redacción sobre el contenido de la memoria debe hacerse en presente y todo lo referido al proyecto en pasado, siempre de modo impersonal.

%----------------------------------------------------------------------------------------

\section{Compilación como herramienta de verificación} % 1 página

%----------------------------------------------------------------------------------------

%----------------------------------------------------------------------------------------

\section{Pruebas unitarias del firmware} % 1 página

%----------------------------------------------------------------------------------------

%----------------------------------------------------------------------------------------

\section{Pruebas de comunicación entre software y firmware} % 1 página

%----------------------------------------------------------------------------------------

%----------------------------------------------------------------------------------------

\section{Pruebas de latencia} % 1 página

%----------------------------------------------------------------------------------------


\include{Chapters/Chapter5}

%----------------------------------------------------------------------------------------
% Apéndices
%----------------------------------------------------------------------------------------

\appendix

% Incluir apéndices desde archivos separados si es necesario
%\include{Appendices/AppendixA}

%----------------------------------------------------------------------------------------
% Bibliografía
%----------------------------------------------------------------------------------------

\renewcommand{\bibname}{Bibliografía} % Para asegurarte de que el título sea correcto
\phantomsection % Necesario para que el enlace del marcador sea correcto

\printbibliography[heading=bibintoc]

\end{document}






