% Chapter 2
\chapter{Introducción específica}

\label{Chapter2}

%----------------------------------------------------------------------------------------

Todos los capítulos deben comenzar con un breve párrafo introductorio que indique cuál es el contenido que se encontrará al leerlo.  La redacción sobre el contenido de la memoria debe hacerse en presente y todo lo referido al proyecto en pasado, siempre de modo impersonal.

%----------------------------------------------------------------------------------------

\section{Introducción a la problemática} % 1 página

%----------------------------------------------------------------------------------------

%----------------------------------------------------------------------------------------

\section{Hardware de desarrollo} % 1 página

%----------------------------------------------------------------------------------------

%----------------------------------------------------------------------------------------

\section{Herramientas de desarrollo de firmware} % 1 página

%----------------------------------------------------------------------------------------

%----------------------------------------------------------------------------------------

\section{Framework para el desarrollo del software} % 1 página

%----------------------------------------------------------------------------------------