% Chapter 2
\chapter{Introducción específica}

\label{Chapter2}

%----------------------------------------------------------------------------------------

En este capítulo se describe brevemente cómo se codifican los mensajes MIDI de acuerdo al estándar, y posteriormente se listan las herramientas utilizadas para el desarrollo del firmware, el software y el hardware del trabajo.

%----------------------------------------------------------------------------------------

\section{Formato de mensajes MIDI} % 1 página

El estándar MIDI está basado en el envío de datos a través de un puerto serie opto-aislado que opera a una velocidad de 31250 baudios. Para codificar los mensajes, en primer lugar se hace la distinción entre dos tipos de bytes: de Estado (\textit{Status Byte}) y de Datos (\textit{Data Byte}). En la figura \ref{fig:midi-byte} se ilustra esta diferencia.

\begin{figure}[ht]
	\centering
	\includegraphics[width=0.5\textwidth]{./Figures/midi-byte.png}
	\caption{Tipos de ``bytes'' según el estándar MIDI.}
	\label{fig:midi-byte}
\end{figure}

Un byte de datos es aquel que tiene su MSB (\textit{Most Significant Bit}, bit más significativo) igual a cero. Es decir que su rango está entre \texttt{0x00} (0 decimal) y \texttt{0x0F} (127 decimal). Por el contrario, un byte de estado tiene su MSB igual a uno, por lo que su rango oscila entre \texttt{0x80} (128 decimal) y \texttt{0xFF} (255  decimal).

El propósito de un byte de estado es el de actuar como encabezado, identificando el tipo de mensaje MIDI. Este consistirá de un byte de estado, seguido de:
\begin{enumerate}
	\item Ningún byte de dato.
	\item Un byte de dato.
	\item Dos bytes de dato.
	\item Múltiple bytes de dato.
\end{enumerate}

En la figura \ref{fig:structure-midi-message} se ilustran estos cuatro posible casos.

\begin{figure}[ht]
	\centering
	\includegraphics[width=\textwidth]{./Figures/structure-midi-message.png}
	\caption{}
	\label{fig:structure-midi-message}
\end{figure}
%----------------------------------------------------------------------------------------

%----------------------------------------------------------------------------------------

\section{Hardware de desarrollo} % 1 página

%----------------------------------------------------------------------------------------

%----------------------------------------------------------------------------------------

\section{Herramientas de desarrollo de firmware} % 1 página

%----------------------------------------------------------------------------------------

%----------------------------------------------------------------------------------------

\section{Framework para el desarrollo del software} % 1 página

%----------------------------------------------------------------------------------------