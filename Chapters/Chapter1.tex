% Chapter 1

\chapter{Introducción general} % Main chapter title

\label{Chapter1} % For referencing the chapter elsewhere, use \ref{Chapter1}
\label{IntroGeneral}

Todos los capítulos deben comenzar con un breve párrafo introductorio que indique cuál es el contenido que se encontrará al leerlo.  La redacción sobre el contenido de la memoria debe hacerse en presente y todo lo referido al proyecto en pasado, siempre de modo impersonal.

%----------------------------------------------------------------------------------------

% Define some commands to keep the formatting separated from the content
\newcommand{\keyword}[1]{\textbf{#1}}
\newcommand{\tabhead}[1]{\textbf{#1}}
\newcommand{\code}[1]{\texttt{#1}}
\newcommand{\file}[1]{\texttt{\bfseries#1}}
\newcommand{\option}[1]{\texttt{\itshape#1}}
\newcommand{\grados}{$^{\circ}$}

%----------------------------------------------------------------------------------------

%\section{Introducción}

%----------------------------------------------------------------------------------------
\section{Introducción a la problemática} % 1 página

Es frecuente encontrar en estudios de grabación y/o producción musical una variedad de instrumentos musicales electrónicos, principalmente sintetizadores o unidades de efectos.

Estos equipos se comunican mediante un estándar denominado MIDI (\textit{Musical Instrument Digital Interface}, Interfaz Digital de Instrumentos Musicales). Éste define un mecanismo de comunicación que permite la transferencia de información entre los instrumentos propiamente dichos, computadoras y otros dispositivos relacionados con la producción musical.

Para garantizar una correcta interoperabilidad, MIDI no sólo define el protocolo (es decir, cómo es el formato de los mensajes), si no también cómo debe ser la conexión física entre ellos mediante la especificación de los cables y conectores que deben usarse.

En la especificación original, la la MMA (\textit{MIDI Manufacturers Association}, Asociacion de Fabricantes MIDI) adoptó el conector DIN-5 de 180° como interfaz física estandarizada para interconectar equipos \cite{mma:midi_spec}. En la figura \ref{fig:din5} se muestra como ejemplo el panel trasero de un sintetizador, donde se ven dos conectores etiquetados como MIDI OUT y MIDI IN.

\begin{figure}[ht]
	\centering
	\includegraphics[width=0.5\textwidth]{./Figures/din-connector.jpg}
	\caption{Ejemplo de cable y conectores DIN-5 pertenecientes al panel trasero de un sintetizador.}
	\label{fig:din5}
\end{figure}

En base a lo anterior se desliza que la comunicación es unidireccional: se requiere de un conector para enviar datos (MIDI OUT) y otro para recibirlos (MIDI IN). El estándar también menciona un puerto adicional (y opcional) llamado MIDI THRU que replica lo recibido por la entrada. Esto permite encadenar varios dispositivos entre sí, creando lo que se conoce como una topología \textit{daisy chain}. En la figura \ref{fig:daisy} se muestra un diagrama que ejemplifica este tipo de conexión.

\begin{figure}[ht]
	\centering
	\includegraphics[width=\textwidth]{./Figures/daisy-chain.png}
	\caption{Diagrama de una topología \textit{daisy chain}. Ésta requiere que el equipo disponga de un puerto MIDI THRU.}
	\label{fig:daisy}
\end{figure}

Al incrementar la cantidad de equipos se dificulta la interconexión entre ellos. En la figura \ref{fig:ad-hoc} se muestra una hipotética conexión de dos controladores con cuatro sintetizadores.

\begin{figure}[ht]
	\centering
	\includegraphics[width=\textwidth]{./Figures/ad-hoc.png}
	\caption{Ejemplo de una conexión hipotética entre dos controladores y cuatro sintetizadores. El secuenciador controlaría a los sintetizadores 2, 3 y 4, mientras que el controlador controlaría al sintetizador 1.}
	\label{fig:ad-hoc}
\end{figure}

Dado que la salida MIDI THRU replica lo que ingresa por MIDI IN, el secuenciador solo es capaz de controlar la cadena de sintetizadores 2, 3 y 4, mientras que el controlador solo podrá controlar al sintetizador 1. Si se desea hacer lo opuesto es necesario reconectar los equipos, hecho que puede resultar incómodo desde un punto de vista operativo e incluso propenso a errores.

La solución a esta clase de problemas consiste en incorporar un equipo adicional llamado \textit{router}, el cual posee un conjunto de entradas y salidas. Al interconectar los equipos queda una topología en estrella, siendo el router el nodo central. En la figura \ref{fig:centralizada} se muestran los mismos equipos que en la figura \ref{fig:ad-hoc} conectados a través del router.

\begin{figure}[ht]
	\centering
	\includegraphics[width=\textwidth]{./Figures/centralizada.png}
	\caption{Reconexión de los equipos de la figura \ref{fig:ad-hoc}, esta vez usando un router central.}
	\label{fig:centralizada}
\end{figure}

Puede observarse que el cableado es más directo e intuitivo. Esto minimiza la probabilidad de errores. Por otro lado, tanto el controlador como el secuenciador son capaces de controlar a cualquiera de los cuatro sintetizadores, dando como resultado mayor flexibilidad de uso.
%----------------------------------------------------------------------------------------

\section{Estado del arte}

Hoy en día existen diversas soluciones comerciales en el mercado global. A la hora de elegir un equipo se tienen en cuenta (entre otras cosas) las siguientes características:
\begin{itemize}
	\item Cantidad de entradas y salidas.
	\item Capacidad de direccionamiento de las entradas.
	\item Filtrado de mensajes MIDI.
	\item Configuración mediante PC o \textit{stand-alone}.
\end{itemize}

En esta sección se presenta una lista general (pero no exhaustiva) de algunas de las alternativas más populares.

\subsection{Kenton THRU-12}
Desarrollado por la firma británica Kenton, este modelo posee una entrada que es replicada a sus doce salidas. A los equipos con esta características se los conoce también como \textit{thru boxes}. En la figura \ref{fig:kenton} se muestra una fotografía del producto.

\begin{figure}[ht]
	\centering
	\includegraphics[width=0.5\textwidth]{./Figures/kenton.png}
	\caption{Kenton THRU-12. Es una \textit{thru box} de una entrada y doce salidas. Imagen tomada de la página oficial del fabricante\protect\footnotemark.}
	\label{fig:kenton}
\end{figure}

\footnotetext{Imagen tomada de \url{https://kentonuk.com/product/thru-12/}}

\subsection{MIDI Solutions Quadra Merge}
Desarrollado por la firma canadiense MIDI Solutions, este modelo posee cuatro entradas que se combinan y son redirigidas a dos salidas. Los equipos que fusionan varias entradas en una o más salidas son conocidos como \textit{merge boxes}. En la figura \ref{fig:midi-solutions} se muestra una fotografía del producto.

\begin{figure}[ht]
	\centering
	\includegraphics[width=0.4\textwidth]{./Figures/midi-solutions.jpg}
	\caption{MIDI Solutions Quadra Merge. Es una \textit{merge box} de cuatro entradas y dos salidas. Imagen tomada de la página oficial del fabricante\protect\footnotemark.}
	\label{fig:midi-solutions}
\end{figure}

\footnotetext{Imagen tomada de \url{https://midisolutions.com/prodqmr.htm}}

\subsection{Conductive Labs MRCC}
Esta solución de la firma estadounidense Conductive Labs posee once entradas y diecisiete salidas. También ofrece opciones avanzadas de ``routeo'' y filtrado de mensajes, configurables a través de una pantalla integrada. En la figura \ref{fig:mrcc} se muestra una fotografía del producto.

\begin{figure}[ht]
	\centering
	\includegraphics[width=0.8\textwidth]{./Figures/mrcc.png}
	\caption{Conductive Labs MRCC. Es un router MIDI de once entradas y diecisiete salidas, con soporte de routeo y filtrado de mensajes. Imagen tomada de la página oficial del fabricante\protect\footnotemark.}
	\label{fig:mrcc}
\end{figure}

\footnotetext{Imagen tomada de \url{https://conductivelabs.com/mrcc/?v=c582dec943ff}}

\subsection{iConnectivity mioXL}
Esta solución de la firma estadounidense iConnectivity posee ocho entradas y doce salidas. También soporta MIDI sobre Ethernet y USB. Al igual que el producto anterior, también ofrece opciones avanzadas de ``routeo'' y filtrado de mensajes configurables desde una aplicación de escritorio. En la figura \ref{fig:mioxl} se muestra una fotografía del producto.

\begin{figure}[ht]
	\centering
	\includegraphics[width=0.9\textwidth]{./Figures/mioxl.jpg}
	\caption{iConnectivity mioXL. Es un router MIDI de ocho entradas y doce salidas, con soporte de routeo y filtrado de mensajes. Imagen tomada de la página oficial del fabricante\protect\footnotemark.}
	\label{fig:mioxl}
\end{figure}

\footnotetext{Imagen tomada de \url{https://www.iconnectivity.com/mioxl}}


\subsection{Comparativa}
Luego de mostrar algunas alternativas comerciales, se recopila en la tabla \ref{tab:comparativa-modelos} la lista de características analizada de cada alternativa, agregando también el costo del producto en dólares estadounidenses.

\begin{table}[h]
	\centering
	\caption[Comparativa entre alternativas comerciales.]{Comparativa de las alternativas comerciales expuestas.}
	\label{tab:comparativa-modelos}
	\resizebox{\textwidth}{!}{%
		\begin{tabular}{lcccccc}
			\toprule
			\textbf{Modelo} & \textbf{Entradas} & \textbf{Salidas} & \textbf{Routeo} & \textbf{Filtrado} & \textbf{Configuración} & \textbf{Costo} \\
			\midrule
			Kenton THRU-12 & 1 & 12 & No & No & - & \$ 150 \\
			MIDI Solutions Quadra Merge & 4 & 2 & No & No & - & \$ 175 \\
			Conductive Labs MRCC & 11 & 17 & Si & Si & \textit{Stand-alone} & \$ 439 \\
			iConnectivity mioXL & 8 & 12 & Si & Si & Aplicación & \$ 500 \\
			\bottomrule
			\hline
		\end{tabular}%
	}
\end{table}

%----------------------------------------------------------------------------------------

\section{Fundamentos y motivaciones del desarrollo propuesto}

%----------------------------------------------------------------------------------------

\section{Objetivos y alcance}

%----------------------------------------------------------------------------------------
