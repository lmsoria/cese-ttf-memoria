% Chapter Template

\chapter{Ensayos y resultados} % Main chapter title

\label{Chapter4} % Change X to a consecutive number; for referencing this chapter elsewhere, use \ref{ChapterX}
Todos los capítulos deben comenzar con un breve párrafo introductorio que indique cuál es el contenido que se encontrará al leerlo.  La redacción sobre el contenido de la memoria debe hacerse en presente y todo lo referido al proyecto en pasado, siempre de modo impersonal.

%----------------------------------------------------------------------------------------

\section{Compilación como herramienta de verificación} % 1 página

%----------------------------------------------------------------------------------------

%----------------------------------------------------------------------------------------

\section{Pruebas unitarias del firmware} % 1 página

%----------------------------------------------------------------------------------------

%----------------------------------------------------------------------------------------

\section{Pruebas de comunicación entre software y firmware} % 1 página

%----------------------------------------------------------------------------------------

%----------------------------------------------------------------------------------------

\section{Pruebas de latencia} % 1 página

%----------------------------------------------------------------------------------------

